%% start of file `template.tex'.
%% Copyright 2006-2013 Xavier Danaux (xdanaux@gmail.com).
%
% This work may be distributed and/or modified under the
% conditions of the LaTeX Project Public License version 1.3c,
% available at http://www.latex-project.org/lppl/.


\documentclass[11pt,a4paper,roman]{moderncv}        % possible options include font size ('10pt', '11pt' and '12pt'), paper size ('a4paper', 'letterpaper', 'a5paper', 'legalpaper', 'executivepaper' and 'landscape') and font family ('sans' and 'roman')

% moderncv themes
\moderncvstyle{classic}                             % style options are 'casual' (default), 'classic', 'oldstyle' and 'banking'
\moderncvcolor{orange}                             % color options 'blue' (default), 'orange', 'green', 'red', 'purple', 'grey' and 'black'
%\renewcommand{\familydefault}{\sfdefault}         % to set the default font; use '\sfdefault' for the default sans serif font, '\rmdefault' for the default roman one, or any tex font name
%\nopagenumbers{}                                  % uncomment to suppress automatic page numbering for CVs longer than one page

% character encoding
%\usepackage[utf8]{inputenc}                       % if you are not using xelatex ou lualatex, replace by the encoding you are using
%\usepackage{CJKutf8}                              % if you need to use CJK to typeset your resume in Chinese, Japanese or Korean

% adjust the page margins
\usepackage[scale=0.75]{geometry}
%\setlength{\hintscolumnwidth}{3cm}                % if you want to change the width of the column with the dates
%\setlength{\makecvtitlenamewidth}{10cm}           % for the 'classic' style, if you want to force the width allocated to your name and avoid line breaks. be careful though, the length is normally calculated to avoid any overlap with your personal info; use this at your own typographical risks...

% personal data
\name{Roland}{Haas}
\title{Curriculum vitae}                           % optional, remove / comment the line if not wanted
\address{National Center for Supercomputing Applications\\
University of Illinois at Urbana-Champaign\\
1205 W. Clark St., MC-257}{Urbana IL 61801}{}% optional, remove / comment the line if not wanted; the "postcode city" and "country" arguments can be omitted or provided empty
\phone[fixed]{+1~(217)~3004228}
\email{rhaas@illinois.edu}                    % optional, remove / comment the line if not wanted
\homepage{www.ncsa.illinois.edu/~rhaas}            % optional, remove / comment the line if not wanted
%\social[linkedin]{rhaas80}                         % optional, remove / comment the line if not wanted
%\social[twitter]{jdoe}                             % optional, remove / comment the line if not wanted
\social[github]{rhaas80}                           % optional, remove / comment the line if not wanted
%\quote{Some quote}                                 % optional, remove / comment the line if not wanted

% to show numerical labels in the bibliography (default is to show no labels); only useful if you make citations in your resume
%\makeatletter
%\renewcommand*{\bibliographyitemlabel}{\@biblabel{\arabic{enumiv}}}
%\makeatother
%\renewcommand*{\bibliographyitemlabel}{[\arabic{enumiv}]}% CONSIDER REPLACING THE ABOVE BY THIS

% bibliography with mutiple entries
%\usepackage{multibib}
%\newcites{book,misc}{{Books},{Others}}
%----------------------------------------------------------------------------------
%            content
%----------------------------------------------------------------------------------
\begin{document}
%\begin{CJK*}{UTF8}{gbsn}                          % to typeset your resume in Chinese using CJK
%-----       resume       ---------------------------------------------------------
\makecvtitle

\section{Personal information}
\cvitem{Name}{Roland Haas}
\cvitem{Date of birth}{May 29\textsuperscript{th} 1980}
\cvitem{Place of birth}{Herbolzheim}
\cvitem{Nationality}{German, permanent resident of Canada}

\section{Experience}
\cventry{07/2016--current}{senior research programmer}{NCSA}{Urbana}{Advisors: Gabrielle Allen, Greg Bauer}{Numerical Relativity}
\cventry{09/2014--07/2016}{Junior scientist / Postdoc}{Albert Einstein Insititute}{Potsdam}{Advisor: Alessandra Buonanno}{Numerical Relativity}
\cventry{09/2011--08/2014}{Postdoctoral research fellow}{Caltech}{Pasadena}{Advisor: Christian Ott}{Numerical Relativity}
\cventry{08/2008--09/2011}{Postdoctoral research fellow}{Georgia Tech}{Atlanta}{Advisor: Pablo Laguna}{Numerical astrophysics}

\section{PhD thesis}
\cvitem{title}{\emph{Self-force on point particles in orbit around a Schwarzschild black hole}}
\cvitem{supervisors}{Eric Poisson, University of Guelph, Canada}
%\cvitem{description}{We examine the motion of a point scalar or
%electromagnetic charge in orbit around a Schwarzschild black hole. Similar to
%the situation in quantum field theory, the field close to the particle
%requires renormalization, separating a finite physical contribution from the
%infinite renormalizable part. We implement a fourth order finite-difference
%scheme to calculate the retarded field. Our code can handle both circular and
%highly eccentric orbits around the black hole. }

\section{Master thesis}
\cvitem{title}{\emph{Mass loss of a scalar charge in cosmological spacetimes}}
\cvitem{supervisors}{Eric Poisson, University of Guelph, Canada}
%\cvitem{description}{In extension to prior work done by Burko, Harte and Poisson the behaviour of a scalar point charge in various cosmological spacetimes is studied. Charges placed on a comoving world line exhibit a time-dependent rest mass. We find two classes of spacetimes, in which the particle shows qualitatively different behaviour. In both classes the particle will initially radiate a part or all of its rest mass away.}

\section{Languages}
\cvitem{German}{native speaker}
\cvitem{English}{fluent}
\cvitem{French}{basic}

\section{Computer skills}
\cvitem{Programming Languages}{Fortran (77 and 90), C, C++, m68k assembly language,
Perl, python, Tcl, awk, shell-scripting, basic html, basic PHP, basic
Javascript, \LaTeX}
\cvitem{Parallel code frameworks}{MPI, OpenMP, basic CUDA}
\cvitem{Application frameworks}{Cactus computational toolkit, SpEC, PETSc}
\cvitem{Version control}{git, subversion, mercurial, darcs, cvs}
\cvitem{Scientific software}{numpy, h5py, matplotlib, gnuplot, VisIt, Paraview, doxygen}
\cvitem{Operating systems}{Linux (Debian, Ubuntu, RedHat), macOS, Windows
(mostly XP, 95), AmigaOS}
\cvitem{Infrastructure}{Deployment and maintenance of
apache-based group website, MediaWiki, centralized git and subversion
repositories including customized web interface for user and repository
management using Submin, setup of mailing lists using mailman and exim, user
account management. Kubernetes, Docker and JupyterHub to run the
Einstein Toolkit Tutorial server. Docker containers in HPC environments using
Shifter.}

\section{Awards}
\cvitem{2010--2012}{NSERC postdoctoral Fellowship  (USD 80,000)}
\cvitem{2006--2008}{NSERC postgraduate scholarship  (CAD 41,000)}
\cvitem{2005}{Ontario Graduate Scholarship  (CAD 15,000)}
\cvitem{2005}{Governor General's Academic Medal {\small\emph{Awarded by the
Governor General to the student graduating with the highest average from a
university program}}}

\section{Memberships}
\cvitem{2014--current}{Association for Computing Machinery}
\cvitem{2014--current}{German Physical Society}
\cvitem{2008--current}{American Physical Society}

\section{Services}
\cvitem{2008--current}{Maintainer of the Einstein Toolkit, a collaborative NSF
funded effort by LSU, NCSA, RIT, Georgia Tech, and Caltech to provide robust
simulation codes for numerical relativity
and numerical astrophysics with 256 registered users at 176 different groups.}
\cvitem{07/2017}{Lead organizer of the EinsteinToolkit workshop at the NCSA /
UIUC meeting in Urbana-Champaign}
\cvitem{2012--2014}{Organizer of the relativity section of the weekly TAPIR seminars.}
\cvitem{07/2013}{Organizer of the Einstein Toolkit Summer Workshop at Caltech,
where all maintainer met to discuss future directions of the project.}
\cvitem{2011}{co-organizer of the EinsteinToolkit workshop at the APS meeting
in Atlanta}
\cvitem{2007--current}{Referee for PRD, PRL, and CQG.}
\cvitem{04/2009}{Session chair Numerical Simulations of Black holes and
Neutron Stars, April APS
meeting, Washington DC.}

\section{Lectures}
\cvlistitem{Tutorial session on using the Einstein Toolkit at the Einstein
Toolkit workshop at RIT, 2019.}
\cvlistitem{Tutorial session on using adaptive mesh refinement at the EU
Einstein Toolkit workshop in Lisbon, Portugal, 2018.}
\cvlistitem{Tutorial session on writing an analysis module at the Spring
Einstein Toolkit workshop
attached to the April APS meeting in Atlanta, 2012.}
\cvlistitem{Tutorial session and introduction to the Einstein Toolkit at the
Summer Einstein Toolkit
workshop at the Caltech Gravitational-Wave Astrophysics School 2013.}

\section{Invited talks}
\cvlistitem{``Gravitational and electromagnetic signatures from the tidal
disruption of stars'', Caltech, Pasadena. CaJAGWR Seminars.}
\cvlistitem{``Three-Dimensional General-Relativistic Hydrodynamic Simulations
of Binary Neutron Star Coalescence and Stellar Collapse with Multipatch
Grids'', UIUC, Urbana-Champaign.  Theoretical Astrophysics and General
Relativity Seminar.}
\cvlistitem{``Community astrophysics science with the Einstein Toolkit'',
Urbana-Champaign. Theoretical Astrophysics and General Relativity Seminar.}

\section{Grants}
\cvlistitem{Principal investigator of NSF XRAC grant TG-PHY160053
``Convergence of Numerical Relativity and Deep Learning for Gravitational Wave
Astrophysics''}
\cvlistitem{Co-principal investigator of NSF XRAC grant TG-PHY100033
``Simulations of Relativistic Astrophysical Systems''}
\cvlistitem{Co-principal investigator of NERSC project m152 ``Central Engine
Models for Core-Collapse Supernovae and Long Gamma-Ray Bursts''}

\section{Students mentored}
\cvlistitem{Jeffrey Kaplan. Graduate Student. Project: binary neutron star inspirals with SpEC.}
\cvlistitem{Jonas Lippuner. Graduate Student. Project: binary neutron star inspirals with SpEC.}
\cvlistitem{Sherwood Richers. Graduate Student. Project: Neutrino Transport in
Supernova Simulations.}
\cvlistitem{Shawn Rosofksy. Graduate Student. Project: binary neutron star
inspirals with Cactus.}
\cvlistitem{Hannah Klion Summer. Undergraduate Research Fellowship (SURF) student in 2012.
Project: Gravitational
Waves from Rapidly Rotating Core-Collapse Supernovae.}
\cvlistitem{Cheol Woo (Peter) Park. Summer Undergraduate Research Fellowship
(SURF) student in 2012. Project: black hole perturbation theory and white
dwarf disruption by an intermediate mass black hole.}
\cvlistitem{Cutter Coryell. Summer Undergraduate Research Fellowship (SURF) student in 2013.
Project: Testing Fully Dynamical Adaptive Mesh Refinement in the Einstein Toolkit.}
\cvlistitem{Dhara Mehta. Undergraduate researcher (SPIN) in 2017. Project:
Automatically prune and archive simulation results produced by the Einstein
Toolkit.}
\cvlistitem{Wei Ren. Undergraduate researcher in 2017. Project: Extrapolating
gravitational waves produced by the Einstein Toolkit to Scri+.}
\cvlistitem{Daniel Johnson. Undergraduate researcher in 2017. Python
Open-source Waveform ExtractoR: An open source, python package to monitor and
post-process numerical relativity simulations.}
\cvlistitem{Nikita Jain. Undergraduate researcher (SPIN) in 2017. Project: A GPU
accelerated BSSN using GAMER.}
\cvlistitem{Pablo Brubeck. Undergraduate research fellow in 2017. Project:
Producing initial data for Cactus using LORENE.}
\cvlistitem{Sibo Wang. Undergraduate researcher (SPIN) in 2017. Project: Using
the Adams-Bashforth timestepper in Cactus.}
\cvlistitem{Vedant Puri. Undergraduate researcher (SPIN) in 2017. Testing
Scheduled Jacobi Relaxation methods for use in the Einstein Toolkit.}
\cvlistitem{Debopam Sanyal. Undergraduate researcher (SPIN) in 2018. Comparing
methods to extrapolate gravitational waves to Scri+}
\cvlistitem{Nicolas White. Undergraduate researcher (INCLUSION) in 2018.
Incorporating the ENIGMA gravitational wave model into LALsuite.}
\cvlistitem{Sarah Habib. Undergraduate researcher (INCLUDION) in 2018, 2019.
Gauge invariant measurement of eccentricity in gravitational waves,
implementing a method to reduce eccentricity in simulations using the Einstein
Toolkit.}
\cvlistitem{Zeran Zhu. Undergraduate researcher (SPIN) in 2018. Generic output
routines for the Einstein Toolkit.}
\cvlistitem{Bing-Jyun (Johnny) Tsao. Undergraduate researcher (SPIN) in 2019.
Solving the Poisson equation on irregular domains.}
\cvlistitem{Brockton Brendal. Undergraduate researcher in 2019. Implementing
methods to extrapolate gravitational waves to Scri+ in the NCSA POWER code.}
\cvlistitem{Bridgette Davey. Undergraduate researcher (INCLUSION) in 2019.
Processing numerical relativity simulation results for use by LIGO.}
\cvlistitem{Joseph Adamo. Undergraduate researcher in 2019. Incorporating the
ENIGMA gravitational wave model into LALsuite.}
\cvlistitem{Keiwen Zhang. Undergraduate researcher in 2019. Improving the
quality of gravitational waves produced using the Einstein Toolkit.}
\cvlistitem{Yufeng Luo. Undergraduate researcher in 2019. DataVault an opens
storage infrastructure for results obtained using the Einstein Toolkit.}

\end{document}


%% end of file `template.tex'.

