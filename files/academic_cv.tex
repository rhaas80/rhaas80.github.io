%% start of file `template.tex'.
%% Copyright 2006-2013 Xavier Danaux (xdanaux@gmail.com).
%
% This work may be distributed and/or modified under the
% conditions of the LaTeX Project Public License version 1.3c,
% available at http://www.latex-project.org/lppl/.


\documentclass[11pt,a4paper,roman]{moderncv}        % possible options include font size ('10pt', '11pt' and '12pt'), paper size ('a4paper', 'letterpaper', 'a5paper', 'legalpaper', 'executivepaper' and 'landscape') and font family ('sans' and 'roman')

% moderncv themes
\moderncvstyle{classic}                             % style options are 'casual' (default), 'classic', 'oldstyle' and 'banking'
\moderncvcolor{orange}                             % color options 'blue' (default), 'orange', 'green', 'red', 'purple', 'grey' and 'black'
%\renewcommand{\familydefault}{\sfdefault}         % to set the default font; use '\sfdefault' for the default sans serif font, '\rmdefault' for the default roman one, or any tex font name
%\nopagenumbers{}                                  % uncomment to suppress automatic page numbering for CVs longer than one page

% character encoding
%\usepackage[utf8]{inputenc}                       % if you are not using xelatex ou lualatex, replace by the encoding you are using
%\usepackage{CJKutf8}                              % if you need to use CJK to typeset your resume in Chinese, Japanese or Korean

% adjust the page margins
\usepackage[scale=0.75]{geometry}
%\setlength{\hintscolumnwidth}{3cm}                % if you want to change the width of the column with the dates
%\setlength{\makecvtitlenamewidth}{10cm}           % for the 'classic' style, if you want to force the width allocated to your name and avoid line breaks. be careful though, the length is normally calculated to avoid any overlap with your personal info; use this at your own typographical risks...

% personal data
\name{Roland}{Haas}
\title{Curriculum vitae}                           % optional, remove / comment the line if not wanted
\address{National Center for Supercomputing Applications\\
University of Illinois at Urbana-Champaign\\
1205 W. Clark St., MC-257}{Urbana IL 61801}{}% optional, remove / comment the line if not wanted; the "postcode city" and "country" arguments can be omitted or provided empty
\phone[fixed]{+1~(217)~300-4228}
\email{rhaas@illinois.edu}                    % optional, remove / comment the line if not wanted
\homepage{www.ncsa.illinois.edu/\textasciitilde{}rhaas}            % optional, remove / comment the line if not wanted
%\social[linkedin]{rhaas80}                         % optional, remove / comment the line if not wanted
%\social[twitter]{jdoe}                             % optional, remove / comment the line if not wanted
\social[github]{rhaas80}                           % optional, remove / comment the line if not wanted
%\quote{Some quote}                                 % optional, remove / comment the line if not wanted

% AASTeX journal abbrevs
\newcommand\aj{\textrm{AJ}}%        % Astronomical Journal 
\newcommand\araa{\textrm{ARA\&A}}%  % Annual Review of Astron and Astrophys 
\newcommand\apj{\textrm{ApJ}}%    % Astrophysical Journal 
\newcommand\apjl{\textrm{ApJL}}     % Astrophysical Journal, Letters 
\newcommand\apjs{\textrm{ApJS}}%    % Astrophysical Journal, Supplement 
\newcommand\ao{\textrm{ApOpt}}%   % Applied Optics 
\newcommand\apss{\textrm{Ap\&SS}}%  % Astrophysics and Space Science 
\newcommand\aap{\textrm{A\&A}}%     % Astronomy and Astrophysics 
\newcommand\aapr{\textrm{A\&A~Rv}}%  % Astronomy and Astrophysics Reviews 
\newcommand\aaps{\textrm{A\&AS}}%    % Astronomy and Astrophysics, Supplement 
\newcommand\azh{\textrm{AZh}}%       % Astronomicheskii Zhurnal 
\newcommand\baas{\textrm{BAAS}}%     % Bulletin of the AAS 
\newcommand\icarus{\textrm{Icarus}}% % Icarus
\newcommand\jaavso{\textrm{JAAVSO}}  % The Journal of the American Association of Variable Star Observers
\newcommand\jrasc{\textrm{JRASC}}%   % Journal of the RAS of Canada 
\newcommand\memras{\textrm{MmRAS}}%  % Memoirs of the RAS 
\newcommand\mnras{\textrm{MNRAS}}%   % Monthly Notices of the RAS 
\newcommand\pra{\textrm{PhRvA}}% % Physical Review A: General Physics 
\newcommand\prb{\textrm{PhRvB}}% % Physical Review B: Solid State 
\newcommand\prc{\textrm{PhRvC}}% % Physical Review C 
\newcommand\prd{\textrm{PhRvD}}% % Physical Review D 
\newcommand\pre{\textrm{PhRvE}}% % Physical Review E 
\newcommand\prl{\textrm{PhRvL}}% % Physical Review Letters 
\newcommand\pasp{\textrm{PASP}}%     % Publications of the ASP 
\newcommand\pasj{\textrm{PASJ}}%     % Publications of the ASJ 
\newcommand\qjras{\textrm{QJRAS}}%   % Quarterly Journal of the RAS 
\newcommand\skytel{\textrm{S\&T}}%   % Sky and Telescope 
\newcommand\solphys{\textrm{SoPh}}% % Solar Physics 
\newcommand\sovast{\textrm{Soviet~Ast.}}% % Soviet Astronomy 
\newcommand\ssr{\textrm{SSRv}}% % Space Science Reviews 
\newcommand\zap{\textrm{ZA}}%       % Zeitschrift fuer Astrophysik 
\newcommand\nat{\textrm{Nature}}%  % Nature 
\newcommand\iaucirc{\textrm{IAUC}}% % IAU Cirulars 
\newcommand\aplett{\textrm{Astrophys.~Lett.}}%  % Astrophysics Letters 
\newcommand\apspr{\textrm{Astrophys.~Space~Phys.~Res.}}% % Astrophysics Space Physics Research 
\newcommand\bain{\textrm{BAN}}% % Bulletin Astronomical Institute of the Netherlands 
\newcommand\fcp{\textrm{FCPh}}%   % Fundamental Cosmic Physics 
\newcommand\gca{\textrm{GeoCoA}}% % Geochimica Cosmochimica Acta 
\newcommand\grl{\textrm{Geophys.~Res.~Lett.}}%  % Geophysics Research Letters 
\newcommand\jcp{\textrm{JChPh}}%     % Journal of Chemical Physics 
\newcommand\jgr{\textrm{J.~Geophys.~Res.}}%     % Journal of Geophysics Research 
\newcommand\jqsrt{\textrm{JQSRT}}%   % Journal of Quantitiative Spectroscopy and Radiative Trasfer 
\newcommand\memsai{\textrm{MmSAI}}% % Mem. Societa Astronomica Italiana 
\newcommand\nphysa{\textrm{NuPhA}}%     % Nuclear Physics A 
\newcommand\physrep{\textrm{PhR}}%       % Physics Reports 
\newcommand\physscr{\textrm{PhyS}}%        % Physica Scripta 
\newcommand\planss{\textrm{Planet.~Space~Sci.}}%  % Planetary Space Science 
\newcommand\procspie{\textrm{Proc.~SPIE}}%      % Proceedings of the SPIE 

\newcommand\actaa{\textrm{AcA}}%  % Acta Astronomica
\newcommand\caa{\textrm{ChA\&A}}%  % Chinese Astronomy and Astrophysics
\newcommand\cjaa{\textrm{ChJA\&A}}%  % Chinese Journal of Astronomy and Astrophysics
\newcommand\jcap{\textrm{JCAP}}%  % Journal of Cosmology and Astroparticle Physics
\newcommand\na{\textrm{NewA}}%  % New Astronomy
\newcommand\nar{\textrm{NewAR}}%  % New Astronomy Review
\newcommand\pasa{\textrm{PASA}}%  % Publications of the Astron. Soc. of Australia
\newcommand\rmxaa{\textrm{RMxAA}}%  % Revista Mexicana de Astronomia y Astrofisica

%% added feb 9, 2016
\newcommand\maps{\textrm{M\&PS}}% Meteoritics and Planetary Science
\newcommand\aas{\textrm{AAS Meeting Abstracts}}% American Astronomical Society Meeting Abstracts
\newcommand\dps{\textrm{AAS/DPS Meeting Abstracts}}% American Astronomical Society/Division for Planetary Sciences Meeting Abstracts


% to show numerical labels in the bibliography (default is to show no labels); only useful if you make citations in your resume
%\makeatletter
%\renewcommand*{\bibliographyitemlabel}{\@biblabel{\arabic{enumiv}}}
%\makeatother
\renewcommand*{\bibliographyitemlabel}{[\arabic{enumiv}]}% CONSIDER REPLACING THE ABOVE BY THIS

% bibliography with mutiple entries
%\usepackage{multibib}
%\newcites{book,misc}{{Books},{Others}}
%----------------------------------------------------------------------------------
%            content
%----------------------------------------------------------------------------------
\begin{document}
%\begin{CJK*}{UTF8}{gbsn}                          % to typeset your resume in Chinese using CJK
%-----       resume       ---------------------------------------------------------
\makecvtitle

\section{Contact information}
\cvitem{Name}{Roland Haas}
\cvitem{Email}{rhaas@illinois.edu}
\cvitem{Web}{https://www.ncsa.illinois.edu/\textasciitilde{}rhaas}
\cvitem{Address}{National Center for Supercomputing Applications\newline
1205 W. Clark St Mail Code 257\newline
University of Illinois Urbana, IL 61801-2311\newline
Phone: 217-300-4228}

\section{Professional Experience}
\cventry{07/2016--current}{Senior research programmer}{National Center for Supercomputing Applications, University of Illinois}{Urbana, IL}{}{Numerical Relativity}
\cventry{09/2014--07/2016}{Junior scientist / Postdoc}{Albert Einstein Insititute}{Potsdam}{Advisor: Alessandra Buonanno}{Numerical Relativity}
\cventry{09/2011--08/2014}{Postdoctoral research fellow}{Caltech}{Pasadena}{Advisor: Christian Ott}{Numerical Relativity}
\cventry{08/2008--09/2011}{Postdoctoral research fellow}{Georgia Tech}{Atlanta}{Advisor: Pablo Laguna}{Numerical astrophysics}

\section{Education}
\cventry{12/2005--08/2008}{PhD in physics}{University of Guelph}{Guelph}{}{
  \emph{Self-force on point particles in orbit around a Schwarzschild black hole}\\
  Advisor: Eric Poisson}
\cventry{09/2003--12/2005}{MSc in physics}{University of Guelph}{Guelph}{}{
  \emph{Mass loss of a scalar charge in cosmological spacetimes}\\
  Advisor: Eric Poisson}
\cventry{09/1999-09/2003}{Diplom in physics}{Universitaet
Konstanz}{Konstanz}{Germany}{unfinished, left for Canada}
\cventry{09/1990 --08/1999}{High school}{Scheffel Gymnasium}{Lahr}{Germany} {}

\pagebreak
\section{Professional Activities}

\cvitem{07/2023}{Lead organizer, Einstein Toolkit CSSI Working Workshop,
Rochester Institute of Technology, Rochester, NY}

\cvitem{07/2023}{Presenter, Einstein Toolkit Summer School,
Rochester Institute of Technology, Rochester, NY}

\cvitem{07/2022}{Lead organizer, Einstein Toolkit CSSI Working Workshop,
University of Illinois, Urbana, IL}

\cvitem{06/2022}{Co-organizer and presenter, Einstein Toolkit Summer School,
University of Idaho, Moscow, ID}

\cvitem{07/2021}{Lead organizer, Einstein Toolkit Summer School, University of
Illinois, Urbana, IL}

\cvitem{08/2020}{Presenter, North American Einstein Toolkit Workshop,
Louisiana State University, Baton Rouge, LA}

\cvitem{09/2019}{Presenter and scientific organization committee member,
European Einstein Toolkit Meeting, King's College London, London, UK}

\cvitem{06/2019}{Presenter, North American Einstein Toolkit workshop,
Rochester Institute of Technology, Rochester, NY}

\cvitem{10/2018}{Co-organizer, Deep Learning for Multimessenger Astrophysics:
Real-time Discovery at Scale workshop, University of Illinois, Urbana, IL}

\cvitem{09/2018}{Presenter, European Einstein Toolkit Workshop,
University of Lisbon, Lisbon, Portugal}

\cvitem{07/2018}{Presenter and scientific organization committee member, 
North American Einstein Toolkit workshop, Georgia Tech, Atlanta, GA}

\cvitem{06/2018}{Presenter and organization committee member, 
Mexican Einstein Toolkit school, Tecnológico de Monterrey, Guadalajaraa}

\cvitem{10/2017}{Presenter, EU Einstein Toolkit 2017 \& EdFest, Universitat de
les Illes Balears, Palma}

\cvitem{07/2017}{Lead organizer of the EinsteinToolkit workshop at the NCSA /
UIUC meeting in Urbana-Champaign}

\cvitem{02/2017}{Participant, Workshop on MHD method in the Einstein Toolkit,
Columbia University, New York, NY}

\cvitem{10/2016}{Observer, ExaHyPE consortium meeting, LRZ,  Garching}

\cvitem{06/2016}{Participant, Einstein Toolkit EU School and Workshop,
University of Trento, Trento}

\cvitem{08/2015}{Participant, ET Workshop 2015, University of Stockholm,
Stockholm}

\cvitem{07/2015}{Local organizing committee member, CGWAS Caltech
Gravitatioanl Wave Astrophysics School, Caltech, Pasadena, CA}

\cvitem{06/2014}{Co-organizer, Capra Meeting on Radiation Reaction in General
Relativity, Caltech, Pasadena, CA}

\cvitem{07/2013}{Organizer of the Einstein Toolkit Summer Workshop at Caltech,
where all maintainer met to discuss future directions of the project.}

\cvitem{07/2013}{Local organizing committee member and presenter, CGWAS Caltech
Gravitatioanl Wave Astrophysics School, Caltech, Pasadena, CA}

\cvitem{2012--2014}{Organizer of the relativity section of the weekly TAPIR
seminars.}

\cvitem{10/2012}{Participant, ET Workshop Fall, Rochester Institute of
Technology, Rochester, NY}

\cvitem{04/2012}{Presenter and co-organizer of the EinsteinToolkit workshop at
the APS meeting in Atlanta}

\cvitem{2011--2020}{Member, LIGO Science Collaboration}

\cvitem{04/2009}{Session chair Numerical Simulations of Black holes and
Neutron Stars, April APS meeting, Washington DC.}

\cvitem{2008--current}{Maintainer of the Einstein Toolkit, a collaborative NSF
funded effort by LSU, NCSA, RIT, Georgia Tech, and Caltech to provide robust
simulation codes for numerical relativity and numerical astrophysics.}

\cvitem{2007--current}{Referee for JOSS, PRD, PRL, and CQG.}

\section{Grants and Awards}
\cvitem{07/2023--current}{Co-Principal investigator of NSF OAC grant 2310548
``Elements: An initial value solver for the era of multi-messenger
astrophysics'' (USD 597,233, 3 yrs)}

\cvitem{04/2020--current}{Principal investigator of NSF OAC grant 2004879
``The Einstein Toolkit ecosystem: Enabling fundamental research in the era of
multi-messenger astrophysics'' (USD 683,514, 4 yrs)}

\cvitem{07/2016--06/2022}{Co-principal investigator of NSF OAC grant 1550514
``Einstein Toolkit Community Integration and Data Exploration'' (USD 450,000, 6 yrs)}

\cvitem{10/2020--current}{Principal investigator of NSF XRAC grant TG-PHY160053
``Convergence of Numerical Relativity and Deep Learning for Gravitational Wave
Astrophysics''}

\cvitem{2010--2012}{NSERC postdoctoral Fellowship  (USD 80,000)}

\cvitem{2006--2008}{NSERC postgraduate scholarship  (CAD 41,000)}

\cvitem{2005}{Ontario Graduate Scholarship  (CAD 15,000)}

\cvitem{2005}{Governor General's Academic Medal {\small\emph{Awarded by the
Governor General to the student graduating with the highest average from a
university program}}}

\section{Research Interests and Areas}
\cvlistitem{Numerical relativity}
\cvlistitem{Relativistic (magneto-)hydrodynamics}
\cvlistitem{Extreme mass ratio inspirals and self-force problems}
\cvlistitem{Black hole perturbation theory}
\cvlistitem{Gravitational and electromagnetic emission from mixed black
hole—star systems}
\cvlistitem{The Einstein Toolkit}
\cvlistitem{Numerical techniques for mesh refinement, and elliptic problems}

\section{Software}
\cvlistitem{``Einstein Toolkit'', \url{https://einsteintoolkit.org}}
\cvlistitem{``Cactus Computational Toolkit'', \url{https://cactuscode.org}}
\cvlistitem{``Carpet mesh refinement driver'',
\url{https://bitbucket.org/eschnett/carpet}}
\cvlistitem{``CarpetX mesh refinement driver'',
\url{https://github.com/eschnett/carpetx}}
\cvlistitem{``Outflow'',
\url{https://github.com/rhaas80/Outflow}}
\cvlistitem{``ReadInterpolate'',
\url{https://github.com/rhaas80/ReadInterpolate}}
\cvlistitem{``GRHydro'',
\url{https://bitbucket.org/einsteintoolkit/einsteinevolve}}
\cvlistitem{``IllinoisGRMHD'',
\url{https://bitbucket.org/zach_etienne/wvuthorns}}
\cvlistitem{``Simulation Factory'',
\url{https://bitbucket.org/simfactory/simfactory2}}
\cvlistitem{``Eccentricity reduction for NR simulations'',
\url{https://github.com/ncsagravity/eccred}}
\cvlistitem{``SpEC'',
\url{https://www.black-holes.org/code/SpEC.html}}
\cvlistitem{``mpitar'',
\url{https://git.ncsa.illinois.edu/rhaas/mpitar}}
\cvlistitem{``parfu'',
\url{https://github.com/ncsa/parfu_archive_tool}}
\cvlistitem{``pa\_volume'',
\url{https://github.com/rhaas80/pa_volume}}
\cvlistitem{Einstein Toolkit Summer School 2021 website,
\url{https://github.com/EinsteinToolkit/et2021uiuc}}
\cvlistitem{Einstein Toolkit CI testing service,
\url{https://github.com/EinsteinToolkit/tests/tree/scripts}}

\section{Students mentored}
\cvlistitem{Jeffrey Kaplan. Graduate Student. Project: binary neutron star inspirals with SpEC.}
\cvlistitem{Jonas Lippuner. Graduate Student. Project: binary neutron star inspirals with SpEC.}
\cvlistitem{Sherwood Richers. Graduate Student. Project: Neutrino Transport in
Supernova Simulations.}
\cvlistitem{Shawn Rosofksy. Graduate Student. Project: binary neutron star
inspirals with Cactus.}
\cvlistitem{Yufeng Luo. Graduate Student. Project: Stability of rotating
neutron stars. University of Wyoming.}
\cvlistitem{Hannah Klion. Summer Undergraduate Research Fellowship (SURF) student in 2012.
Project: Gravitational
Waves from Rapidly Rotating Core-Collapse Supernovae.}
\cvlistitem{Cheol Woo (Peter) Park. Summer Undergraduate Research Fellowship
(SURF) student in 2012. Project: black hole perturbation theory and white
dwarf disruption by an intermediate mass black hole.}
\cvlistitem{Cutter Coryell. Summer Undergraduate Research Fellowship (SURF) student in 2013.
Project: Testing Fully Dynamical Adaptive Mesh Refinement in the Einstein Toolkit.}
\cvlistitem{Dhara Mehta. Undergraduate researcher (SPIN) in 2017. Project:
Automatically prune and archive simulation results produced by the Einstein
Toolkit.}
\cvlistitem{Wei Ren. Undergraduate researcher in 2017. Project: Extrapolating
gravitational waves produced by the Einstein Toolkit to Scri+.}
\cvlistitem{Daniel Johnson. Undergraduate researcher in 2017. Python
Open-source Waveform ExtractoR: An open source, python package to monitor and
post-process numerical relativity simulations.}
\cvlistitem{Nikita Jain. Undergraduate researcher (SPIN) in 2017. Project: A GPU
accelerated BSSN using GAMER.}
\cvlistitem{Pablo Brubeck. Undergraduate research fellow in 2017. Project:
Producing initial data for Cactus using LORENE.}
\cvlistitem{Sibo Wang. Undergraduate researcher (SPIN) in 2017. Project: Using
the Adams-Bashforth timestepper in Cactus.}
\cvlistitem{Vedant Puri. Undergraduate researcher (SPIN) in 2017. Testing
Scheduled Jacobi Relaxation methods for use in the Einstein Toolkit.}
\cvlistitem{Debopam Sanyal. Undergraduate researcher (SPIN) in 2018. Comparing
methods to extrapolate gravitational waves to Scri+}
\cvlistitem{Nicolas White. Undergraduate researcher (INCLUSION) in 2018.
Incorporating the ENIGMA gravitational wave model into LALsuite.}
\cvlistitem{Sarah Habib. Undergraduate researcher (INCLUSION) in 2018, 2019.
Gauge invariant measurement of eccentricity in gravitational waves,
implementing a method to reduce eccentricity in simulations using the Einstein
Toolkit.}
\cvlistitem{Zeran Zhu. Undergraduate researcher (SPIN) in 2018. Generic output
routines for the Einstein Toolkit.}
\cvlistitem{Bing-Jyun (Johnny) Tsao. Undergraduate researcher (SPIN) in 2019.
Solving the Poisson equation on irregular domains.}
\cvlistitem{Brockton Brendal. Undergraduate researcher in 2019. Implementing
methods to extrapolate gravitational waves to Scri+ in the NCSA POWER code.}
\cvlistitem{Bridgette Davey. Undergraduate researcher (INCLUSION) in 2019.
Processing numerical relativity simulation results for use by LIGO.}
\cvlistitem{Joseph Adamo. Undergraduate researcher in 2019. Incorporating the
ENIGMA gravitational wave model into LALsuite.}
\cvlistitem{Kaiwen Zhang. Undergraduate researcher in 2019. Improving the
quality of gravitational waves produced using the Einstein Toolkit.}
\cvlistitem{Yufeng Luo. Undergraduate researcher in 2019. DataVault an opens
storage infrastructure for results obtained using the Einstein Toolkit.}
\cvlistitem{Robert Nagel. Undergraduate researcher in 2020. Constructing a
uniform framework to characterize numerical relativity waveforms.}
\cvlistitem{Nuocheng Pan. Undergraduate researcher in 2020. Improving the
robustness of the ENIGMA waveform generation code.}
\cvlistitem{Mohammed Jamil. Undergraduate researcher in 2020. Using GitHub
action for continuous integration testing of the Einstein Toolkit.}
\cvlistitem{Mingxin Li. Undergraduate researcher in 2021. Improving
performance of characterizing numerical relativity waveforms.}
\cvlistitem{Parth Tiyagi. Undergraduate researcher in 2021. On the fly
training data generation for large scale artificial neural network based
gravitational waveform searches.}
\cvlistitem{Hrishikesh Kalyanaraman. Undergraduate researcher in 2022.
Continuous integration testing for the Einstein Toolkit.}
\cvlistitem{Nadine Kuo. Undergraduate researcher in 2022.
Continuous integration testing for the Einstein Toolkit.}
\cvlistitem{Lisa Leung. Undergraduate researcher in 2022.
Improving quality of Einstein Toolkit binary black hole simulations.}
\cvlistitem{Ashley Turner. Undergraduate researcher in 2023.
Improving quality of Einstein Toolkit binary black hole simulations.}
\cvlistitem{Chirag Maheshwari. Undergraduate researcher in 2023.
Improving quality of Einstein Toolkit binary black hole simulations.}

\section{Lectures}
\cvlistitem{Tutorial session on using the Einstein Toolkit at the Einstein
Toolkit workshop at Rochester, 2023.}
\cvlistitem{Tutorial session on using the Einstein Toolkit at the Einstein
Toolkit workshop at Dublin, 2022.}
\cvlistitem{Tutorial session on using the Einstein Toolkit at the Numerical
Relativity Summer School at ICERM, 2022.}
\cvlistitem{Tutorial session on using the Einstein Toolkit at the Einstein
Toolkit workshop at UIdaho, 2022.}
\cvlistitem{Tutorial session on using the Einstein Toolkit at the Einstein
Toolkit workshop at LSU, 2020.}
\cvlistitem{Tutorial session on using the Einstein Toolkit at the Einstein
Toolkit workshop at RIT, 2019.}
\cvlistitem{Lecture on using MPI and OpenMP at the PIRE Winter School at the
University of Arizona, 2018.}
\cvlistitem{Tutorial session on using adaptive mesh refinement at the EU
Einstein Toolkit workshop in Lisbon, Portugal, 2018.}
\cvlistitem{Tutorial session on writing an analysis module at the Spring
Einstein Toolkit workshop attached to the April APS meeting in Atlanta, 2012.}
\cvlistitem{Tutorial session and introduction to the Einstein Toolkit at the
Summer Einstein Toolkit
workshop at the Caltech Gravitational-Wave Astrophysics School 2013.}

\section{Invited talks}
\cvlistitem{``Gravitational and electromagnetic signatures from the tidal
disruption of stars'', Caltech, Pasadena, CA. CaJAGWR Seminars. April, 2012.}
\cvlistitem{``Three-Dimensional General-Relativistic Hydrodynamic Simulations
of Binary Neutron Star Coalescence and Stellar Collapse with Multipatch
Grids'', UIUC, Urbana-Champaign, IL.  Theoretical Astrophysics and General
Relativity Seminar. April, 2013. }
\cvlistitem{``Core collapse and binary neutron star inspiral
simulation using multipatch grids'',
University of Southampton, Southampton, UK. Gravity Seminar.
February, 2015.}
\cvlistitem{``Update on binary neutron star merger simulations'',
Max-Planck-Institute for Gravitational Physics, Golm, Germany. AEI Seminar.
April, 2016. }
\cvlistitem{``Simulating multi-physics astrophysical problems using
current and future codes'',
Leibnitz Rechenzentrum, Garching, Germany. ExaHYPe collaboration meeting.
April, 2017. }
\cvlistitem{``Update on binary neutron star merger simulations'',
Goethe-University, Frankfurt, Germany. Astro coffee.
April, 2017. }
\cvlistitem{``Community astrophysics science with the Einstein Toolkit'',
UIUC, Urbana-Champaign, IL. Theoretical Astrophysics and General Relativity Seminar.
September, 2017. }
\cvlistitem{``Assessing confidence in numerical relativity waveforms of
binary neutron star mergers'', Nikhef, Amsterdam, Netherlands. Seminar talk.
September, 2018. }
\cvlistitem{``Present and future astrophysics simulations using the Einstein
Toolkit'', University of Wyoming, WY,  Colloquium presentation.  December,
2021. }
\cvlistitem{``Using computational physics to study relativity and more'',
University of Guelph, ON,  Colloquium presentation.  September, 2023. }

\newpage
\section{Contributed talks and posters}
\cvlistitem{``HydroOpenMPToy status'', King's College London, London, UK.
Einstein Toolkit Workshop. September, 2019}

\cvlistitem{``The NCSA eccentric gravitational waveform
catalog'', Denver, Colorado. APS April Meeting. April, 2019}

\cvlistitem{``The NCSA eccentric gravitational waveform
catalog'', Denver, CO. APS April Meeting. April, 2019}

\cvlistitem{``BOSS-LDG using Blue Waters for LIGO data analysis'',  Columbus,
OH. APS April Meeting. April, 2018}

\cvlistitem{``Assessing confidence in numerical relativity waveforms of
binary neutron star mergers'',  Columbus, Ohio. APS April Meeting. April, 2018}

\cvlistitem{``HydroOpenMPToy status'', University des Illes Baleares, Palma,
Spain. Einstein Toolkit Workshop. October 2017.}

\cvlistitem{``Neutron star simulations with SpEC'', University of Stockholm,
Stockholm, Sweden. MICRA meeting. August, 2015}

\cvlistitem{``Postprocrocessing data in Cactus'', University of Stockholm,
Stockholm, Sweden. Einstein Toolkit Workshop. June, 2015}

\cvlistitem{``Binary Neutron Star simulations using SpEC'', University of
Thessaloniki, Thessaloniki, Greece. Workshop on Binary Neutron Star Mergers.
May, 2015}

\cvlistitem{``Binary Neutron Star simulations using SpEC'', UCSD, San Diego,
CA. Pacific Coast Gravity Meeting. March, 2014}

\cvlistitem{``Binary Neutron Star simulations using SpEC'', Savannah, GA. APS
April Meeting. April, 2014}

\cvlistitem{``Self-force driven inspiral of a scalar point
particle into a Schwarzschild black hole'', UCLA, Los Angeles, CA. TASC
Meeting 2013. December, 2013}

\cvlistitem{``Binary NS simulations using SpEC'', Uniwersytet Warszawski,
Warsaw, Poland. Amaldi meeting. July, 2013}

\cvlistitem{``Binary NS simulations using SpEC'', Denver, CO. APS April
Meeting. April, 2013}

\cvlistitem{``Binary NS simulations using SpEC'', UCD, Davis, CA. Pacific
Coast Gravity Meeting. March, 2013}

\cvlistitem{``Binary NS simulations using SpEC'', Carnegie Observatories,
Pasadena, CA. TASC Meeting 2012. November, 2012}

\cvlistitem{``Self-force driven inspiral of a scalar point
particle into a Schwarzschild black hole: a
progress report'', University of Maryland, Collega Park, MD. Capra Meeting on
Radiation Reaction in General Relativity. June, 2012}

\cvlistitem{``Progress report: Binary NS simulations using SpEC'', Atlanta,
GA. APS April Meeting. April, 2012}

\cvlistitem{``Self-force driven inspiral of a scalar point particle into a
Schwarzschild black hole'', UCSB, Sata-Barbara, CA. Pacific Coast Gravity
Meeting. March, 2012}

\cvlistitem{``Gravitational and Electromagnetic Signatures from the Tidal
Disruption of a White Dwarf by an Intermediate Mass Black Hole'', Florida
Atlantic University. Boca Raton, FL. Gulf Coast Gravity Meeting. May, 2011}

\cvlistitem{``Gravitational and Electromagnetic Signatures from the Tidal
Disruption of a White Dwarf by an Intermediate Mass Black Hole'', Anaheim, CA.
APS April Meeting. May, 2011}

\cvlistitem{``Progress Report: Gravitational and Electromagnetic Signatures
from Tidal Disruption of a White Dwarf by an Intermediate Mass Black Hole'',
North Carolina State University, Raleigh, NC. Eastern Gravity Meeting. May,
2010}

\cvlistitem{``Gravitational and Electromagnetic Signatures from the Tidal
Disruption of a White Dwarf by an Intermediate Mass Black Hole'', Washington,
DC. APS April Meeting. February, 2010}

\cvlistitem{``Black Hole - Neutron Star Binary Simulations at Georgia Tech'',
Louisiana State University, Baton Rouge, LA. Gulf Coast Gravity Meeting.
April, 2009}

\cvlistitem{``Electromagnetic self-force for eccentric orbits in Schwarzschild
spacetime'', CNRS, Orleans, France. Capra meeting on Radiation Reaction, June,
2008}

\cvlistitem{``Electromagnetic self-force for eccentric orbits in Schwarzschild
spacetime: A progress report'', St. Louis University, St Louis, MO. Midwest
Relativity Meeting, November, 2007}

\cvlistitem{``Scalar self-force for eccentric orbits in Schwarzschild
spacetime'', University of Alabama, Huntsville, AL. Capra meeting on Radiation
Reaction, June, 2007}

\cvlistitem{``Scalar self-force for eccentric orbits in Schwarzschild
spacetime'',  University of New Brunswick, Fredericton, NB. Canadian
Conference on General Relativity and Relativistic Astrophysics, May, 2007}

\cvlistitem{``Scalar self-force for eccentric orbits in Schwarzschild
spacetime'', Washington University, St Louis, MO. Midwest Relativity Meeting,
November, 2006}

\begin{thebibliography}{99}
% https://ui.adsabs.harvard.edu/user/libraries/pzWebybvThCQyfOW6CVxrA
% format used:
% %ZEncoding:latex%ZLinelength:0\bibitem{%R} %5.3l, ``%T,'' %j, %V, %p (%Y).
\item[] \begin{tabular}{l@{\hspace{4ex}}r@{\hspace{4ex}}r@{\hspace{4ex}}r}
		&\href{https://www.scopus.com/authid/detail.uri?authorId=36084420600}{Scopus}
		&\href{https://scholar.google.com/citations?user=rZj1mR8AAAAJ}{Google Scholar}
		&\ldots since 2018\\
\hline
Citations	&4063	&6015	&4319\\
h-index		&61	&40	&35\\
i10-index	&N/A	&56	&54\\
\end{tabular}
\subsection{Short author list papers}
\bibitem{2025CQGra..42b5016K} Kalinani, J.~V., Ji, L., Ennoggi, L., et al., ``AsterX: a new open-source GPU-accelerated GRMHD code for dynamical spacetimes,'' Classical and Quantum Gravity, 42, 025016 (2025).
\bibitem{2024Symm...16..343L} \underline{Luo, Y.}, Tsokaros, A., Haas, R., \& Ury{\={u}}, K., ``General Relativistic Stability and Gravitational Wave Content of Rotating Triaxial Neutron Stars,'' Symmetry, 16, 343 (2024).
\bibitem{2024Symm...16..343L} \underline{Luo, Y.}, Tsokaros, A., Haas, R., \& Ury{\={u}}, K., ``General Relativistic Stability and Gravitational Wave Content of Rotating Triaxial Neutron Stars,'' Symmetry, 16, 343 (2024).
\bibitem{2024PNAS..12111888P} Park, H., Patel, P., Haas, R., \& Huerta, E.~A., ``APACE: AlphaFold2 and advanced computing as a service for accelerated discovery in biophysics,'' Proceedings of the National Academy of Science, 121, e2311888121 (2024).
\bibitem{2024LRR....27....3K} Kumar, R., Dexheimer, V., Jahan, J., et al., ``Theoretical and experimental constraints for the equation of state of dense and hot matter,'' Living Reviews in Relativity, 27, 3 (2024).
\bibitem{2024CQGra..41b5002L} \underline{Luo, Y.}, Zhang, Q., Haas, R., Etienne, Z.~B., \& Allen, G., ``HPC-driven computational reproducibility in numerical relativity codes: a use case study with IllinoisGRMHD,'' Classical and Quantum Gravity, 41, 025002 (2024).
\bibitem{2024ApJ...961L..26C} Curtis, S., Bosch, P., M{\"o}sta, P., et al., ``Magnetized Outflows from Short-lived Neutron Star Merger Remnants Can Produce a Blue Kilonova,'' \apjl, 961, L26 (2024).
\bibitem{2023PhRvD.107j3055F} Foucart, F., Duez, M.~D., Haas, R., et al., ``General relativistic simulations of collapsing binary neutron star mergers with Monte Carlo neutrino transport,'' \prd, 107, 103055 (2023).
\bibitem{2023PhRvD.107f4038J} Joshi, A.~V., Rosofsky, S.~G., Haas, R., \& Huerta, E.~A., ``Numerical relativity higher order gravitational waveforms of eccentric, spinning, nonprecessing binary black hole mergers,'' \prd, 107, 064038 (2023).
\bibitem{2023PhRvD.107d4037W} Werneck, L.~R., Etienne, Z.~B., Murguia-Berthier, A., et al., ``Addition of tabulated equation of state and neutrino leakage support to IllinoisGRMHD,'' \prd, 107, 044037 (2023).
\bibitem{2023CQGra..40t5009S} Shankar, S., M{\"o}sta, P., Brandt, S.~R., et al., ``GRaM-X: a new GPU-accelerated dynamical spacetime GRMHD code for Exascale computing with the Einstein Toolkit,'' Classical and Quantum Gravity, 40, 205009 (2023).
\bibitem{2022MNRAS.512.1499R} ($*$) Radice, D., Bernuzzi, S., Perego, A., \& Haas, R., ``A new moment-based general-relativistic neutrino-radiation transport code: Methods and first applications to neutron star mergers,'' \mnras, 512, 1499 (2022).
\bibitem{2021PhRvD.103h4018C} \underline{Chen, Z.}, Huerta, E.~A., Adamo, J., et al., ``Observation of eccentric binary black hole mergers with second and third generation gravitational wave detector networks,'' \prd, 103, 084018 (2021).
\bibitem{2021CoPhC.26708081Z} Zhang, X., Achilles, S., Winkelmann, J., et al., ``Solving the Bethe-Salpeter equation on massively parallel architectures,'' Computer Physics Communications, 267, 108081 (2021).
\bibitem{2021CQGra..38m5016L} \underline{Luo, Y.}, Haas, R., Zhang, Q., \& Allen, G., ``DataVault: a data storage infrastructure for the Einstein Toolkit,'' Classical and Quantum Gravity, 38, 135016 (2021).
\bibitem{2021CQGra..38m5008T} \underline{Tsao, B.-J.}, Haas, R., \& Tsokaros, A., ``Source term method for binary neutron stars initial data,'' Classical and Quantum Gravity, 38, 135008 (2021).
\bibitem{2021CQGra..38l5007H} \underline{Habib, S.}, Ramos-Buades, A., Huerta, E.~A., et al., ``Initial data and eccentricity reduction toolkit for binary black hole numerical relativity waveforms,'' Classical and Quantum Gravity, 38, 125007 (2021).
\bibitem{2021ApJ...919...82W} Wei, W., Huerta, E.~A., Yun, M., et al., ``Deep Learning with Quantized Neural Networks for Gravitational-wave Forecasting of Eccentric Compact Binary Coalescence,'' \apj, 919, 82 (2021).
\bibitem{2020ascl.soft04003E} Etienne, Z.~B., Paschalidis, V., Haas, R., Moesta, P., \& Shapiro, S.~L., ``IllinoisGRMHD: GRMHD code for dynamical spacetimes,'' Astrophysics Source Code Library, ascl:2004.003 (2020).
\bibitem{2020PhRvD.102d4055O} ($*$) Ossokine, S., Buonanno, A., Marsat, S., et al., ``Multipolar effective-one-body waveforms for precessing binary black holes: Construction and validation,'' \prd, 102, 044055 (2020).
\bibitem{2020PhRvD.101d4053V} Vincent, T., Foucart, F., Duez, M.~D., et al., ``Unequal mass binary neutron star simulations with neutrino transport: Ejecta and neutrino emission,'' \prd, 101, 044053 (2020).
\bibitem{2020ApJ...901L..37M} ($*$) M{\"o}sta, P., Radice, D., Haas, R., Schnetter, E., \& Bernuzzi, S., ``A Magnetar Engine for Short GRBs and Kilonovae,'' \apjl, 901, L37 (2020).
\bibitem{2019PhRvD.100f4003H} Huerta, E.~A., Haas, R., Habib, S., et al., ``Physics of eccentric binary black hole mergers: A numerical relativity perspective,'' \prd, 100, 064003 (2019).
\bibitem{2019PhRvD.100d4025R} \underline{Rebei, A.}, Huerta, E.~A., Wang, S., et al., ``Fusing numerical relativity and deep learning to detect higher-order multipole waveforms from eccentric binary black hole mergers,'' \prd, 100, 044025 (2019).
\bibitem{2019PhRvD..99d4008F} Foucart, F., Duez, M.~D., Hinderer, T., et al., ``Gravitational waveforms from spectral Einstein code simulations: Neutron star-neutron star and low-mass black hole-neutron star binaries,'' \prd, 99, 044008 (2019).
\bibitem{2019NatRP...1..600H} Huerta, E.~A., Allen, G., Andreoni, I., et al., ``Enabling real-time multi-messenger astrophysics discoveries with deep learning,'' Nature Reviews Physics, 1, 600 (2019).
\bibitem{2019CSBS....3....5H} Huerta, E.~A., Haas, R., Jha, S., Neubauer, M., \& Katz, D.~S., ``Supporting High-Performance and High-Throughput Computing for Experimental Science,'' Computing and Software for Big Science, 3, 5 (2019).
\bibitem{2018PhRvD..97h3014H} Hossein Nouri, F., Duez, M.~D., Foucart, F., et al., ``Evolution of the magnetized, neutrino-cooled accretion disk in the aftermath of a black hole-neutron star binary merger,'' \prd, 97, 083014 (2018).
\bibitem{2018PhRvD..97h3014H} Hossein Nouri, F., Duez, M.~D., Foucart, F., et al., ``Evolution of the magnetized, neutrino-cooled accretion disk in the aftermath of a black hole-neutron star binary merger,'' \prd, 97, 083014 (2018).
\bibitem{2018PhRvD..97b4031H} ($*$) Huerta, E.~A., Moore, C.~J., Kumar, P., et al., ``Eccentric, nonspinning, inspiral, Gaussian-process merger approximant for the detection and characterization of eccentric binary black hole mergers,'' \prd, 97, 024031 (2018).
\bibitem{2018CQGra..35b7002J} Johnson, D., Huerta, E.~A., \& Haas, R., ``Python Open source Waveform ExtractoR (POWER): an open source, Python package to monitor and post-process numerical relativity simulations,'' Classical and Quantum Gravity, 35, 027002 (2018).
\bibitem{2018ApJ...864..171M} ($*$) M{\"o}sta, P., Roberts, L.~F., Halevi, G., et al., ``r-process Nucleosynthesis from Three-dimensional Magnetorotational Core-collapse Supernovae,'' \apj, 864, 171 (2018).
\bibitem{2018ApJ...855L...3O} ($*$) Ott, C.~D., Roberts, L.~F., da Silva Schneider, A., et al., ``The Progenitor Dependence of Core-collapse Supernovae from Three-dimensional Simulations with Progenitor Models of 12-40 M $_{{\ensuremath{\odot}}}$,'' \apjl, 855, L3 (2018).
\bibitem{2017PhRvL.119q1103F} Fedrow, J.~M., Ott, C.~D., Sperhake, U., et al., ``Gravitational Waves from Binary Black Hole Mergers inside Stars,'' \prl, 119, 171103 (2017).
\bibitem{2017PhRvD..95b4038H} ($*$) Huerta, E.~A., Kumar, P., Agarwal, B., et al., ``Complete waveform model for compact binaries on eccentric orbits,'' \prd, 95, 024038 (2017).
\bibitem{2016PhRvD..94d9903T} Tacik, N., Foucart, F., Pfeiffer, H.~P., et al., ``Erratum: Binary neutron stars with arbitrary spins in numerical relativity [Phys. Rev. D 92, 124012 (2015)],'' \prd, 94, 049903 (2016).
\bibitem{2016PhRvD..93l4062H} Haas, R., Ott, C.~D., Szilagyi, B., et al., ``Simulations of inspiraling and merging double neutron stars using the Spectral Einstein Code,'' \prd, 93, 124062 (2016).
\bibitem{2016PhRvD..93d4064B} Barkett, K., Scheel, M.~A., Haas, R., et al., ``Gravitational waveforms for neutron star binaries from binary black hole simulations,'' \prd, 93, 044064 (2016).
\bibitem{2016PhRvD..93d4019F} ($*$) Foucart, F., Haas, R., Duez, M.~D., et al., ``Low mass binary neutron star mergers: Gravitational waves and neutrino emission,'' \prd, 93, 044019 (2016).
\bibitem{2016ApJ...831...98R} ($*$) Roberts, L.~F., Ott, C.~D., Haas, R., et al., ``General-Relativistic Three-Dimensional Multi-group Neutrino Radiation-Hydrodynamics Simulations of Core-Collapse Supernovae,'' \apj, 831, 98 (2016).
\bibitem{2016ApJ...820...76R} ($*$) Radice, D., Ott, C.~D., Abdikamalov, E., et al., ``Neutrino-driven Convection in Core-collapse Supernovae: High-resolution Simulations,'' \apj, 820, 76 (2016).
\bibitem{2015PhRvD..92l4012T} Tacik, N., Foucart, F., Pfeiffer, H.~P., et al., ``Binary neutron stars with arbitrary spins in numerical relativity,'' \prd, 92, 124012 (2015).
\bibitem{2015PhRvD..91l4021F} ($*$) Foucart, F., O'Connor, E., Roberts, L., et al., ``Post-merger evolution of a neutron star-black hole binary with neutrino transport,'' \prd, 91, 124021 (2015).
\bibitem{2015Natur.528..376M} ($*$) M{\"o}sta, P., Ott, C.~D., Radice, D., et al., ``A large-scale dynamo and magnetoturbulence in rapidly rotating core-collapse supernovae,'' \nat, 528, 376 (2015).
\bibitem{2015CQGra..32q5009E} ($*$) Etienne, Z.~B., Paschalidis, V., Haas, R., M{\"o}sta, P., \& Shapiro, S.~L., ``IllinoisGRMHD: an open-source, user-friendly GRMHD code for dynamical spacetimes,'' Classical and Quantum Gravity, 32, 175009 (2015).
\bibitem{2015ApJ...808...70A} Abdikamalov, E., Ott, C.~D., Radice, D., et al., ``Neutrino-driven Turbulent Convection and Standing Accretion Shock Instability in Three-dimensional Core-collapse Supernovae,'' \apj, 808, 70 (2015).
\bibitem{2014PhRvD..90b4026F} ($*$) Foucart, F., Deaton, M.~B., Duez, M.~D., et al., ``Neutron star-black hole mergers with a nuclear equation of state and neutrino cooling: Dependence in the binary parameters,'' \prd, 90, 024026 (2014).
\bibitem{2014CQGra..31a5005M} ($*$) M{\"o}sta, P., Mundim, B.~C., Faber, J.~A., et al., ``GRHydro: a new open-source general-relativistic magnetohydrodynamics code for the Einstein toolkit,'' Classical and Quantum Gravity, 31, 015005 (2014).
\bibitem{2014ApJ...785L..29M} ($*$) M{\"o}sta, P., \underline{Richers, S.}, Ott, C.~D., et al., ``Magnetorotational Core-collapse Supernovae in Three Dimensions,'' \apjl, 785, L29 (2014).
\bibitem{2013PhRvL.111o1101R} ($*$) Reisswig, C., Ott, C.~D., Abdikamalov, E., et al., ``Formation and Coalescence of Cosmological Supermassive-Black-Hole Binaries in Supermassive-Star Collapse,'' \prl, 111, 151101 (2013).
\bibitem{2013PhRvD..88h4021V} Vega, I., Wardell, B., Diener, P., Cupp, S., \& Haas, R., ``Scalar self-force for eccentric orbits around a Schwarzschild black hole,'' \prd, 88, 084021 (2013).
\bibitem{2013PhRvD..87f4023R} Reisswig, C., Haas, R., Ott, C.~D., et al., ``Three-dimensional general-relativistic hydrodynamic simulations of binary neutron star coalescence and stellar collapse with multipatch grids,'' \prd, 87, 064023 (2013).
\bibitem{2013PhRvD..87d1501Z} Zimmerman, P., Vega, I., Poisson, E., \& Haas, R., ``Self-force as a cosmic censor,'' \prd, 87, 041501 (2013).
\bibitem{2013ApJ...769...85S} Shcherbakov, R.~V., Pe'er, A., Reynolds, C.~S., et al., ``GRB060218 as a Tidal Disruption of a White Dwarf by an Intermediate-mass Black Hole,'' \apj, 769, 85 (2013).
\bibitem{2013ApJ...768..115O} $(*)$ Ott, C.~D., Abdikamalov, E., M{\"o}sta, P., et al., ``General-relativistic Simulations of Three-dimensional Core-collapse Supernovae,'' \apj, 768, 115 (2013).
\bibitem{2012PhRvD..86b4026O} ($*$) Ott, C.~D., Abdikamalov, E., O'Connor, E., et al., ``Correlated gravitational wave and neutrino signals from general-relativistic rapidly rotating iron core collapse,'' \prd, 86, 024026 (2012).
\bibitem{2012CQGra..29w2002H} ($*$) Healy, J., Bode, T., Haas, R., et al., ``Late inspiral and merger of binary black holes in scalar-tensor theories of gravity,'' Classical and Quantum Gravity, 29, 232002 (2012).
\bibitem{2012CQGra..29k5001L} ($*$) L{\"o}ffler, F., Faber, J., Bentivegna, E., et al., ``The Einstein Toolkit: a community computational infrastructure for relativistic astrophysics,'' Classical and Quantum Gravity, 29, 115001 (2012).
\bibitem{2012ApJ...749..117H} ($*$) Haas, R., Shcherbakov, R.~V., Bode, T., \& Laguna, P., ``Tidal Disruptions of White Dwarfs from Ultra-close Encounters with Intermediate-mass Spinning Black Holes,'' \apj, 749, 117 (2012).
\bibitem{2012ApJ...744...45B} Bode, T., Bogdanovi{\'c}, T., Haas, R., et al., ``Mergers of Supermassive Black Holes in Astrophysical Environments,'' \apj, 744, 45 (2012).
\bibitem{2011CQGra..28i4020B} Bogdanovi{\'c}, T., Bode, T., Haas, R., Laguna, P., \& Shoemaker, D., ``Properties of accretion flows around coalescing supermassive black holes,'' Classical and Quantum Gravity, 28, 094020 (2011).
\bibitem{2010ApJ...715.1117B} ($*$) Bode, T., Haas, R., Bogdanovi{\'c}, T., Laguna, P., \& Shoemaker, D., ``Relativistic Mergers of Supermassive Black Holes and Their Electromagnetic Signatures,'' \apj, 715, 1117 (2010).
\bibitem{2008PhDT........26H} Haas, R., ``Self-force on point particles in orbit around a Schwarzschild black hole,'' Ph.D. Thesis, (2008).
\bibitem{2007PhRvD..75l4011H} Haas, R., ``Scalar self-force on eccentric geodesics in Schwarzschild spacetime: A time-domain computation,'' \prd, 75, 124011 (2007).
\bibitem{2006PhRvD..74d4009H} Haas, R., \& Poisson, E., ``Mode-sum regularization of the scalar self-force: Formulation in terms of a tetrad decomposition of the singular field,'' \prd, 74, 044009 (2006).
\bibitem{2005CQGra..22S.739H} Haas, R., \& Poisson, E., ``Mass change and motion of a scalar charge in cosmological spacetimes,'' Classical and Quantum Gravity, 22, S739 (2005).

\subsection{Preprints}
\bibitem{2025arXiv250411537S} Shankar, S., M{\"o}sta, P., Haas, R., \& Schnetter, E., ``3D full-GR simulations of magnetorotational core-collapse supernovae on GPUs: A systematic study of rotation rates and magnetic fields,'' arXiv e-prints, arXiv:2504.11537 (2025).
\bibitem{2025arXiv250309629J} Ji, L., Haas, R., Zlochower, Y., et al., ``GPU-accelerated Subcycling Time Integration with the Einstein Toolkit,'' arXiv e-prints, arXiv:2503.09629 (2025).
\bibitem{2024arXiv241207836K} Kacmaz, S., Haas, R., \& Huerta, E.~A., ``Machine learning-driven conservative-to-primitive conversion in hybrid piecewise polytropic and tabulated equations of state,'' arXiv e-prints, arXiv:2412.07836 (2024).
\bibitem{2024arXiv240906837C} Cruz-Camacho, N., Kumar, R., Reinke Pelicer, M., et al., ``Phase Stability in the 3-Dimensional Open-source Code for the Chiral mean-field Model,'' arXiv e-prints, arXiv:2409.06837 (2024).
\bibitem{2024arXiv240713168T} Tian, M., Gao, L., Dylan Zhang, S., et al., ``SciCode: A Research Coding Benchmark Curated by Scientists,'' arXiv e-prints, arXiv:2407.13168 (2024).
\bibitem{2019arXiv190109967M} Markakis, C.~M., O'Boyle, M.~F., Glennon, D., et al., ``Time-symmetry, symplecticity and stability of Euler-Maclaurin and Lanczos-Dyche integration,'' arXiv e-prints, arXiv:1901.09967 (2019).
\bibitem{2011arXiv1112.3707H} Haas, R., ``Time domain calculation of the electromagnetic self-force on eccentric geodesics in Schwarzschild spacetime,'' arXiv e-prints, arXiv:1112.3707 (2011).

\subsection{Collaboration papers}
\bibitem{2025arXiv250207902R} Reinke Pelicer, M., Cruz-Camacho, N., Conde, C., et al., ``Building Neutron Stars with the MUSES Calculation Engine,'' arXiv e-prints, arXiv:2502.07902 (2025).
\bibitem{2018ApJ...859...47A} Arzoumanian, Z., Baker, P.~T., Brazier, A., et al., ``The NANOGrav 11 Year Data Set: Pulsar-timing Constraints on the Stochastic Gravitational-wave Background,'' \apj, 859, 47 (2018).
\bibitem{2016PhRvX...6d1014A} Abbott, B.~P., Abbott, R., Abbott, T.~D., et al., ``Improved Analysis of GW150914 Using a Fully Spin-Precessing Waveform Model,'' Physical Review X, 6, 041014 (2016).
\bibitem{2016PhRvD..93l2004A} Abbott, B.~P., Abbott, R., Abbott, T.~D., et al., ``Observing gravitational-wave transient GW150914 with minimal assumptions,'' \prd, 93, 122004 (2016).
\bibitem{2013CQGra..31b5012H} Hinder, I., Buonanno, A., Boyle, M., et al., ``Error-analysis and comparison to analytical models of numerical waveforms produced by the NRAR Collaboration,'' Classical and Quantum Gravity, 31, 025012 (2013).

\subsection{Conference papers}
\bibitem{2021HUST.5805629} Steffen, C., Haas, R., Kendig, K., Mainzer, L., Chui, R., \& Fliege, C., ``Efficient Software for Archiving and Retrieving Results of Massive Bioinformatics Analyses in High-Performance Computing Environments,'', 8th International Workshop on HPC User Support Tools (2021).
\bibitem{2019ACM.3332186.3333042} Cupp, S., Brandt, S.~R. \& Haas, R., ``The Presync project: Synchronization automation in the cactus framework'', in ``Practice and Experience in Advanced Research Computing 2019: Rise of the Machines (Learning)'', Chicgo, IL, PEARC '19 (2019).
\bibitem{2018arXiv180800556B} Belkin, M., Haas, R., Arnold, G.~W., et al., ``Container solutions for HPC Systems: A Case Study of Using Shifter on Blue Waters,'' PEARC '18: Proceedings of Practice and Experience in Advanced Research Computing, July 22--26, 2018, Pittsburgh, PA, USA, arXiv:1808.00556 (2018).
\bibitem{2017CUG.159s2} Bauer, G., Anisimov, V., Arnold, G., Bode, B., Cortese, T., Haas, R., Kot, A., Kramer, W., Kwack, J., Li, J., Mendes, C., Mokos, R., \& Steffen, C., ``Updating the SPP Benchmark Suite for Extreme-Scale Systems,'' Redmond WA, Cray User Group Meeting (CUG-2017), (2017).
\bibitem{2017arXiv170908767H} Huerta, E.~A., Haas, R., Fajardo, E., et al., ``BOSS-LDG: A Novel Computational Framework that Brings Together Blue Waters, Open Science Grid, Shifter and the LIGO Data Grid to Accelerate Gravitational Wave Discovery,'' 2017 IEEE 13th International Conference on e-Science (e-Science), arXiv:1709.08767 (2017).

\subsection{White papers}
\bibitem{2023arXiv231101300L} LISA Consortium Waveform Working Group, Afshordi, N., Ak{\c{c}}ay, S., et al., ``Waveform Modelling for the Laser Interferometer Space Antenna,'' arXiv e-prints, arXiv:2311.01300 (2023).
\bibitem{2019arXiv190200522A} Allen, G., Andreoni, I., Bachelet, E., et al., ``Deep Learning for Multi-Messenger Astrophysics: A Gateway for Discovery in the Big Data Era,'' arXiv e-prints, arXiv:1902.00522 (2019).

\subsection{Software products}
\bibitem{2024zndo..12588764B} Brandt, S.~R., Haas, R., Diener, P., et al., ``The Einstein Toolkit (Lev Landau),'' Zenodo:12588764, (2024).
\bibitem{2024zndo..14193969H} Haas, R., Rizzo, M., Boyer, D., et al., ``The Einstein Toolkit (Annie Jump Cannon),'' Zenodo:14193969, (2024).
\bibitem{2023zndo..10380404C} Cupp, S., Brandt, S.~R., Bozzola, G., et al., ``The Einstein Toolkit (Lise Meitner),'' Zenodo:10380404, (2023).
\bibitem{2023zndo...7942541W} Werneck, L., Cupp, S., Assump{\c{c}}{\~a}o, T., et al., ``The Einstein Toolkit (Karl Schwarzschild),'' Zenodo:7942541, (2023).
\bibitem{2022zndo...7245853H} Haas, R., Cheng, C.-H., Diener, P., et al., ``The Einstein Toolkit (Sophie Kowalevski),'' Zenodo:7245853, (2022).
\bibitem{2022zndo...6588641Z} Zlochower, Y., Brandt, S.~R., Diener, P., et al., ``The Einstein Toolkit (Bernhard Riemann),'' Zenodo:6588641, (2022).
\bibitem{2022zndo...6131529S} Schnetter, E., Brandt, S., Cupp, S., et al., ``CarpetX,'' Zenodo, (2022).
\bibitem{2021zndo...5770803B} Brandt, S.~R., Bozzola, G., Cheng, C.-H., et al., ``The Einstein Toolkit (Katherine Johnson),'' Zenodo:5770803, (2021).
\bibitem{2021zndo...4884780E} Etienne, Z., Brandt, S.~R., Diener, P., et al., ``The Einstein Toolkit (Lorentz),'' Zenodo:4884780, (2021).
\bibitem{2020zndo...4298887H} Haas, R., Brandt, S.~R., Gabella, W.~E., et al., ``The Einstein Toolkit (DeWitt-Morette),'' Zenodo:4298887, (2020).
\bibitem{2020zndo...3866075B} Brandt, S.~R., Brendal, B., Gabella, W.~E., et al., ``The Einstein Toolkit (Turing),'' Zenodo:3866075, (2020).
\bibitem{2019zndo...3522086B} Babiuc-Hamilton, M., Brandt, S.~R., Diener, P., et al., ``The Einstein Toolkit (Mayer),'' Zenodo:3522086, (2019).
\bibitem{2018ascl.soft07021J} Johnson, D., Brendal, B., Huerta, E.~A., \& Haas, R., ``POWER: Python Open-source Waveform ExtractoR,'' Astrophysics Source Code Library, ascl:1807.021 (2018).

\end{thebibliography}

\end{document}


%% end of file `template.tex'.

