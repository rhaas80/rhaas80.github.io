\documentclass[12pt]{article}
\usepackage[T1]{fontenc}
\usepackage{lmodern}
\usepackage[nohead,margin=1in]{geometry}
%\usepackage{mathptmx}
\usepackage{amsmath}
\usepackage{amssymb}
\usepackage{url}
\usepackage{color}
\usepackage{aas_macros}

\pagestyle{plain}

\newcommand{\sun}{{\mathord\odot}}
\newcommand{\code}[1]{\textsc{#1}}
\newcommand{\aligo}{aLIGO}
\newcommand{\ligo}{LIGO}
\newcommand{\virgo}{VIRGO}
\newcommand{\kagra}{KAGRA}
\newcommand{\lisa}{eLISA/NGO}
\newcommand{\ET}{Einstein Toolkit}
\newcommand{\tapir}{TAPIR}
\newcommand{\cactus}{\code{Cactus}}
\newcommand{\todo}[1]{{\color{blue}TODO: #1}}
%\renewcommand{\emph}[1]{\textbf{#1}}

\begin{document}
\title{Statement of Research Interests}
\author{Roland Haas}
\date{\today}
\maketitle

\section{Research interests}
My main research interests lie in the field of computational astrophysics,
working on the generation of gravitational waves by
astrophysical systems and the analysis of these waves. My research interests
include aspects of analytical
relativity in particular black hole perturbation theory as well as
semi-analytical aspects thereof. I am strongly interested in computer science,
especially numerical techniques applicable to general relativistic
magnetohydrodynamics. 

I have worked on the gravitational and electromagnetic signatures of the
disruption of white dwarfs by intermediate mass black
holes~\cite{2012arXiv1212.4837S, 2012ApJ...749..117H},
the signature of gas present in the final merger phase of
supermassive black holes~\cite{2012ApJ...744...45B,2010ApJ...715.1117B}, extreme mass ratio
inspirals~\cite{2011arXiv1112.3707H,2007PhRvD..75l4011H,2006PhRvD..74d4009H,2005CQGra..22S.739H,2013PhRvD..87d1501Z},
binary neutron star mergers~\cite{Haas:nsns}, core
collapse supernovae~\cite{2012arXiv1210.6674O,2012PhRvD..86b4026O}, 
and developed multi-patch, adaptive mesh
refinement codes for
numerical magnetohydrodynamics~\cite{2012arXiv1212.1191R}. I am actively
involved in developing and supporting the free, open-source Einstein
Toolkit~\cite{2012CQGra..29k5001L}.

We are at the brink of being able to directly detect the gravitational waves
predicted by General Relativity. With the advanced Laser Interferometer
Gravitational-Wave Observatory (\aligo{}) scheduled to come
on-line in 2014~\cite{advLIGO:Web}, there is great need to model the sources of
gravitational waves that \ligo{} will detect. 
Only by comparing to model waveforms will we be able to extract the
gravitational wave signal from \ligo{'s} detector noise.
These detections, together with
electromagnetic observations of similar events, promise to offer answers to
some of the most exciting questions in astrophysics. Are neutron star --
neutron star mergers and black hole -- neutron star mergers sources for short
hard gamma ray bursts?  How does the unknown equation of state of nuclear
matter affect the merger and what microphysics governs matter at
super-nuclear densities? Are nuclear accretion disks sites of $r$-process
nucleosynthesis? All of these questions require that the system be accurately
modelled and simulated using state of the art, highly parallel codes that
included complex multi-physics combining a general relativistic description of
gravity with a realistic description of matter and radiation. 

I have initiated collaborations both locally with researchers at Caltech and
with researches at remote institutions to tackle these problems. These
collaborations have been very fruitful in bringing together the varied talents
that are necessary to develop the codes and technologies required for
realistic simulations. 

During my career I had the opportunity to collaborate with and co-advise
graduate and undergraduate students and introduce them to scientific methods.
I am currently collaborating with several grad-students at Caltech, being the
primary contact person for one of them.  

I am Co-PI to XSEDE grants of 9.5 million service hours and NERSC grants of
6.5 million service hours. I am involved in the SXS collaboration's funding
efforts to
secure and continue funding from NASA, the Fairchild foundation and NSF. I was
funded by a 2 year NSERC fellowship during 2010 -- 2012.

\subsection{Binary neutron star mergers as sources of gravitational radiation}
% NSNS review by Matt: http://adsabs.harvard.edu/abs/2010CQGra..27k4002D
% NSNS LRR review by Josh: http://adsabs.harvard.edu/abs/2012LRR....15....8F
Double binary neutron star systems are among the expected sources of
gravitational waves detectable by
\ligo{}, \virgo{}, and the Kamioka Gravitational Wave Detector
(\kagra{})~(\cite{Anderson:2007km} and references therein).
Such a double binary neutron star system forms out of a binary
star system that
survives throughout the evolution and supernova explosion of both of its
constituent stars~\cite{Tauris:2003pf}. After formation the tight binary
constantly emits predominantly quadrupolar gravitational waves. Eventually
gravitational wave emission becomes the
dominant source of orbital decay and drives the inspiral. To date this has been
observed only indirectly for example 
in the orbital decay of the Hulse-Taylor pulsar~\cite{Hulse:1975}. 
Finally the neutron stars collide and either form
a temporary hypermassive neutron star or directly collapse to a black
hole (\cite{Faber:2012rw} and references therein). During the final inspiral
phase the system emits a
chirp-like gravitational wave signal which increases in both frequency and
amplitude as the stars approach each other. Once the stars come into
contact the sinusoidal gravitational wave abruptly cuts off 
while the combined object settles into an almost
spheroidal
shape. The massive (neutron star) remnant may then continue to emit
gravitational waves as hydrodynamic instabilities redistribute matter to form a
non-vanishing quadrupole moment until sufficient angular momentum has been
shed via gravitational wave emission to allow collapse to a black 
hole~\cite{Giacomazzo:2011cv}.  Finally the
black hole settles down through quasinormal ringing. 

The inspiral signal encodes information about the mass of the neutron star,
while information about the neutron star equation of state is mostly
accessible in the
very late
inspiral and early merger signal. Detecting the signal and in particular
extracting
system parameters using a matched filtering approach however requires an
accurate model and numerical simulation of the inspiral and merger
process~\cite{Anderson:2007km}. This is usually done in a two step process,
where a few accurate but expensive full numerical simulations are used to
calibrate a secondary set of semi-analytic templates~\cite{Bernuzzi:2012ci}.
This is done as an
efficiency measure because of the large cost associated with full
simulations. Generally the secondary templates are more faithful if longer
numerical waveforms are used to calibrate them. 

%The numerical waveforms are not directly used in the filter pipeline as it is
%\todo{jargon}
%not feasible to cover the search space sufficiently densely with templates
%using fully numerical waveforms. Instead waveforms from numerical relativity
%are used to calibrate semi-analytical waveform templates computed for example
%in the effective one body approach~\cite{Bernuzzi:2012ci}. These approximate
%templates can
%be generated much more efficiently and are used to densely conver the
%search space.

Recent work~\cite{Hinderer:2009ca,Read:2009yp} showed that the equation of
state of neutron star
matter has measurable influence on the tidal deformability of the neutron
stars and through this leaves an imprint the the gravitational wave signal.
Any template used for parameter extraction thus has to take the 
equation of state of neutron star matter into account. 
Unfortunately many current simulations~\cite{Bernuzzi:2012ci,
Giacomazzo:2010bx,Baiotti:2011am,East:2012ww,Foucart:2012vn,Haas:nsns}
assume
a simplified $\Gamma$-law equation of state which is appropriate only during
the inspiral phase. On the other hand those simulations that use a more realistic
nuclear equation of state are very
short~\cite{Hotokezaka:2011dh}, spanning only $5$ -- $7$ orbits.
It is expected~\cite{Read:2009yp} that longer numerical simulations will allow
for a more accurate estimation of equation of state parameters.

In the simulating extreme spacetimes (SXS) collaboration~\cite{SXS:web}
I am simulating the inspiral and merger of binary neutron
stars.
Using \code{SpEC}, we achieve high accuracy over long inspirals 
by employing a hybrid scheme
that uses efficient spectral methods to solve for the smooth spacetime part of
the Einstein-matter system and a finite volume method to solve for the matter
variables.

We are currently capable of simulation $\gtrsim 22$ orbits which opens the
window to infer information about the nuclear equation of state from the late
inspiral signal.

I am implementing
adaptive mesh refinement in the code, increasing the resolution in the crucial
central region of the simulation
domain where the merging neutron stars are located to further increase this
accuracy.

\subsection{Binary neutron star mergers as progenitors of short gamma ray burst
central engines} 
The merger of double neutron star binaries and the accretion of a remnant disk
on the central merger remnant is a leading
candidate for the central engine in short gamma ray
bursts~\cite{Tauris:2003pf}.
During merger the
neutron star material is heated and the cold equation of state 
used in most current simulations becomes an increasingly poorer approximation
of the equation of state of the neutron stars.
This has the
possibility of affecting the predicted lifetime of a hypermassive merger remnant and the
geometry of the surrounding disk. The effect of varying the equation of state 
was first studied
in~\cite{Oechslin:2006uk}, \cite{Sekiguchi:2011zd}
and~\cite{Hotokezaka:2013mm}.
\cite{Oechslin:2006uk} compared results of simulations using different hot
equations of state, using the conformally flat
approximation to General Relativity, while \cite{Sekiguchi:2011zd} employed
full numerical relativity but only a single hot equation of
state. Very recently~\cite{Hotokezaka:2013mm} employed both general
relativity, several (cold) equations of state and simulated inspirals lasting
10 orbits.

Once the stars have coalesced into a single object a jet of
relativistic particles can be ejected along the spin axis.
If the central object is a black hole, then strong magnetic fields
driven by the magnetorotational instability in
the disk extract energy from the spin of the black hole to launch the
jet~\cite{Blandford1977}.
An alternative jet mechanism drives the jet by neutrino annihilation that
deposits energy via pair production above the poles and
provides energy in this manner. The strong neutrino flux
facilitates the formation of r-process elements in the disk and may
drive a neutrino-rich disk wind~\cite{1999ApJ...525L.121F,Roberts:2010wh}.
In both the central black hole and central hypermassive
neutron star case, the magnetic field will collimate outflowing
material along the magnetic field lines.
% Rosswog: http://dx.doi.org/10.1086/312343 (NSNS)
% Roberts: http://dx.doi.org/10.1088/0004-637X/722/1/954 (CCSNe)

%Thus
%binary neutron star mergers, together with core-collapse supernovae, are
%cosmic incubators for nuclei heavier than $^{56}\mathrm{Fe}$.

To facilitate this research I and collaborators are currently adding a more
realistic neutrino
transport scheme, as well as support for magnetohydrodynamics to our existing
adaptive mesh refined multi-patch code~\cite{all:mpmhdnu}. This is a major
step towards being able to simulate jet formation and disk evolution in a
fully relativistic simulation. These simulations will significantly advance
our understanding of the engine behind gamma ray bursts, pinning down the
observational signatures of the various proposed source models. 

\subsection{Core collapse supernovae}
When a massive main sequence star nears the end of its lifetime, it burns
lighter elements to iron in a thermonuclear reaction and forms a growing 
iron core. Once the core reaches
a critical mass, nuclear statistical equilibrium favours dissociation into
$\alpha$-particles and free nucleons and the core begins to collapse in a
runaway process.
As the core collapses its density rises to nuclear densities where the
equation of state suddenly stiffens providing a new hydrostatic equilibrium.
The core however overshoots this equilibrium position and bounces back into
the still infalling material. This launches a shock into the infalling
material that is ultimately responsible for the supernova
explosion. As the shock travels outwards, it dissipates energy via
dissociation of heavy nuclei  as they fall through the shock front and
via escaping neutrinos produced in electron capture reactions. Eventually
the shock stalls and a successful explosion requires a means to revive the
shock. 

Currently there are two commonly accepted mechanisms for shock revival:
(a) the neutrino mechanism~\cite{1985nuas.conf..422W} %,1990RvMP...62..801B,
%1985nuas.conf..422W,2007PhR...442...38J}
where the shock is revived through the energy deposited by
neutrinos escaping from the proto-neutron star core in a gain layer under the
shock front and (b) the magnetohydrodynamic mechanism which operates by
converting the rotational energy of the core into magnetic field energy which
drives the explosion through magnetic
pressure~\cite{Janka:2012wk,Burrows:07b}.
%http://arxiv.org/abs/1206.2503v1
%all in Ott et all supernovae paper

Modern 1d spherical studies~\cite{2001PhRvD..63j4003L} %,2002A&A...396..361R,
%2003ApJ...592..434T, 2012ApJS..199...17S, 2006A&A...450..345K, 2010PhRvL.104y1101H}
using full Boltzmann neutrino transport have shown that in spherical symmetry
neutrino heating
is not sufficient to revive the shock.
A secondary mechanism is required
to prolong the residency time of the infalling material in the gain region for
it to absorb sufficient energy to revive the shock.
Two possible mechanism have been observed in 2D axisymmetric
simulations: the neutrino-driven convection~\cite{1994ApJ...435..339H} %,
%1995ApJ...450..830B, 1996A&A...306..167J, 2000ApJ...541.1033F}
which creates a negative entropy gradient in
the gain region, thus promoting convective overturn of the material,
and the standing accretion shock instability
(SASI)~\cite{2003ApJ...584..971B} %, 2006ApJ...641.1018O, 2006A&A...457..963S,
%2008ApJ...688.1159M, 2008ApJ...685.1069O, 2009ApJ...694..664M,
%2012ApJ...761...72M, 2008ApJ...678.1207I}
which leads to a large scale, sloshing type motion along the axis
of symmetry, also increasing the residence time of material in the gain
region. 

In a recent paper~\cite{2012arXiv1210.6674O} we described a
core collapse simulation simulating collapse and
protoneutron star formation, and that for the first time employed all of full 3d numerics,
high
resolution, neutrino transport and a fully general relativistic
treatment of gravity. This is in contrast to previous
studies~\cite{Kuroda:2012nc,Nordhaus:2010uk,Mueller:2012ak} that
employed a combination of a Newtonian approximation to General Relativity,
simplified neutrino
treatment, lower resolution or were using axisymmetric codes. Our work improves
tremendously on these aspects, we extract, for the first time, 
gravitational waves from full 3d general relativistic collapse and post-bounce
core-collapse supernova simulations. We also verify the relative importance of
the standing accretion shock instability and convective instabilities in 3d
Cartesian simulations. 

The simulation pushed the boundaries of what our
simulation framework was able to handle both in size of the simulation as
well as complexity of the physics involved, yielding output files of terabyte
size that had to be postprocessed off-line to explore the dynamics of the
core-collapse supernova engine.

We are now working on improving our
neutrino transport scheme by implementing a 2 moment, energy dependent scheme
as well as incorporating ideal magnetohydrodynamics into our code. These
improvements will allow us to make quantitative predictions of the neutrino
signal to be expected from the next galactic supernova. 

Having included magnetohydrodynamics we will simulate rapidly rotating core collapse
supernovae as well, thereby obtaining prediction for the neutrino and
gravitational wave signal for this revival mechanism as well. Provided the
neutrino signal of the two explosion channels is as distinctive as the
gravitational wave signal~\cite{2012arXiv1210.6674O} observations of
gravitational waves and/or neutrino signals may well allow us to distinguishing
between the mechanisms.

\subsection{Open source software development}

The codes used
for general relativistic magnetohydrodynamics simulations
have grown beyond the point where a single researcher (or even research group)
can
develop them. Instead large collaborations bringing together physicists,
computer scientists and applied mathematicians have formed to develop and
maintain the
codes~\cite{2012CQGra..29k5001L,SXS:web,astrocodelib:web}.

Free, open source licensed software enables researchers to more easily
share their work and combine efforts.
%
%Making codes available to the public not
%only furthers experimentation with the code by users, thus leading to the
%discovery of new applications for existing code, it can also increase a
%code's reliability. Each new user adds another pair of eyes looking for
%potential bugs, and indeed publishing code along with simulation results is
%often the only way to enable verification of results of numerical
%simulations. 
%
This avoids duplication of effort freeing scientists to concentrate on science
rather than tool-building. 
%
Several groups have in the past compared results of their vacuum
and hydrodynamics codes
for identical physical setups~\cite{Babiuc:2007vr,
Baiotti:2010ka}. Such comparisons, while useful, are hampered if codes are
private and results cannot be reproduced. Open source software is in
line with open science practices such as providing access to raw data from
experiments and peer review of results. 

From a practical point of view, open source software more easily allows for a
separation of concerns, where computer scientists supply the computational
infrastructure, and physicists the physics modules building on these.
Different modules, developed by independent groups can benefit from using
existing components~\cite{2012CQGra..29k5001L}. Postdocs and graduate students that
move from one group to another do not lose access to codes they helped
develop and can continue to contribute to science without interruption..
New groups can more easily join the field if there are basic software
components available, leading to a richer science as more scenarios are
explored using the code. 

I am strongly involved in the \ET{}
collaboration~\cite{2012CQGra..29k5001L} which aims to
provide an open source, stable, well maintained and documented code for 
general relativistic magnetohydrodynamics simulations. The \ET{}
is based on the \code{Cactus} framework and used by groups at Caltech,
LSU, AEI, GaTech, RIT and others for their numerical simulations. The toolkit
currently has 91 registered users from 44 institutions around the world, with
regular releases of updates and code fixes every 6 months. 

Indeed the core-collapse simulations we performed at Caltech are based on the
toolkit and requirements of these simulations are a major driving factor in
the development of the toolkit.

Together with Christian Reisswig (Caltech) I worked on the publicly available
multi-patch code \code{Llama}~\cite{Pollney:2009yz}.
\code{Llama} provides a multi-patch framework for general relativistic
magnetohydrodynamics which covers the simulation domain with multiple patches
adapted to the symmetry of the problem. Typically a central Cartesian block
capable of adaptive mesh refinement is surrounded by a set of inflated
cube~\cite{Thornburg:2004dv} patches with logarithmic
spacing in the radial coordinate and constant angular resolution. Such a
patch system is ideally suited to compact binary and core-collapse simulations
where the central region benefits from mesh refinement and where spherical
or cylindrical symmetry is not a good approximation. In the outer wave zone on
the other hand, spherical symmetry is a good approximation to the system
dynamics and the use of a coordinate system adapted to this symmetry greatly
improves speed and accuracy of the simulation. 

\subsection{Extreme mass ratio inspirals}
Mass segregation
effects concentrate stellar mass black holes and other heavy objects  near the
central massive black holes~\cite{1997MNRAS.284..318S,2000ApJ...545..847M} in
galaxy cores. %,
%2006ApJ...649...91F,2006ApJ...645.1152H}
From there, scattering by
other stars~\cite{2007CQGra..24R.113A}, 
tidal disruption events~\cite{2005ApJ...631L.117M} and resonant
relaxation~\cite{2006ApJ...645L.133H} %,2007MNRAS.379.1083G}
moves compact objects onto eccentric trajectories along which the objects
enters
the strong field region around the central black hole 
where gravitational wave emission dominates the orbital decay. 

A possible future \lisa{} detector will detect such extreme mass ratio
inspirals of a solar mass compact object into a supermassive black hole.
During the
inspiral \lisa{} will
map the black hole spacetime
in exquisite detail~\cite{AmaroSeoane:2012je}. 
These mappings provide a test for the
black hole uniqueness theorems of General Relativity as well as the chance to
directly measure the mass of the supermassive black holes in the
centre of galaxies~\cite{AmaroSeoane:2012je} to very high accuracy.
Just as the
waveforms contain information about the central object so do they encode the
orbital parameters, and detection of extreme mass ratio inspiral signals thus offers
a unique opportunity to learn about the dynamics governing dense stellar
clusters. 

%Due to the large mass difference between the primary's mass 
%and the secondary's mass, these systems are amenable to black hole
%perturbation techniques. 
%In black hole perturbation theory the influence the
%smaller object is treated as a
%perturbation on top of the background spacetime of the supermassive black hole.
%The inspiral is then described as a second order non-linear effect of the
%smaller object interacting with the radiation it itself sources using a
%combination of analytic and semi-analytic techniques.
%Our understanding of the mechanism driving the
%inspiral has greatly improved by developing practical methods to evaluate the
%so called self-force responsible for it~\cite{Barack:01, Barack:2007we}
%following pioneering work by Mino, Sasaki, Tanaka, Quinn and
%Wald~\cite{mino:97,quinn:97}.
%On the theoretical side, our understanding of the second
%order perturbation theory required to predict the inspiral signal has improved
%considerably with the work of~\cite{Pound:2012dk,Gralla:2012db}.

%Recently~\cite{Huerta:2012zy} worked on extending
%the methods developed for extreme mass ratio inspirals to intermediate mass
%ratio systems, where the mass ratio is of order $10^{-3}$ and even to
%comparable mass ratio systems. These studies aim at improving available 
%template banks and reducing the need for time consuming fully generic
%numerical simulations. 
%
There are currently efforts underway to compute
inspiral waveforms using either full 3d simulations or by interpolating the
self-force driving the inspiral from a
table of force values computed along
geodesics~\cite{Diener:2011cc, Warburton:2011fk}.

\bibliographystyle{hunsrt}
\bibliography{refs,rhaaspub}

\end{document}
